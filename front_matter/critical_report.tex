\documentclass[parskip=full]{scrreprt}

\RedeclareSectionCommand[pagestyle=plain,afterskip=10pt plus 1pt]{chapter}
\setkomafont{chapter}{\mdseries\headingfont\fontsize{40}{40}\selectfont\color{black!80}}
\setkomafont{pageheadfoot}{\normalsize}

\def\pnumbox#1{#1\hspace*{9cm}}
\DeclareTOCStyleEntry[
  indent=0pt,
  entrynumberformat=\textcolor{oldred},
  numwidth=2em,
  linefill=\hfill,
  pagenumberbox=\pnumbox,
  pagenumberformat=\itshape
]{tocline}{section}

\usepackage[english]{babel}

\usepackage{fontspec}

\setmainfont{Source Sans Pro}[
  ItalicFont = Source Sans Pro Italic,
  BoldFont = Source Sans Pro Bold,
  BoldItalicFont = Source Sans Pro Bold Italic,
  FontFace = {lt}{n}{Source Sans Pro Light},
  FontFace = {lt}{it}{Source Sans Pro Light Italic},
  FontFace = {sb}{n}{Source Sans Pro Semibold},
  FontFace = {sb}{it}{Source Sans Pro Semibold Italic},
  Numbers = Proportional,
  Ligatures = Common
]
\DeclareRobustCommand{\ltseries}{\fontseries{lt}\selectfont}
\DeclareRobustCommand{\sbseries}{\fontseries{sb}\selectfont}
\DeclareTextFontCommand{\textlt}{\ltseries}
\DeclareTextFontCommand{\textsb}{\sbseries}

\newfontfamily\headingfont{Fredericka the Great}

\usepackage[pass]{geometry}
\newgeometry{twoside,inner=20mm,outer=40mm,top=20mm,bottom=40mm}

\usepackage{url}
\urlstyle{same}

\usepackage{microtype}
\microtypesetup{verbose=silent}

\usepackage{booktabs,array,longtable}
\newcolumntype{L}[1]{>{\raggedright\let\newline\\\arraybackslash\hspace{0pt}}p{#1}}

\usepackage{graphicx}

\usepackage{xcolor}
\definecolor{oldred}{rgb}{.8313,0,0}

\usepackage{pdfpages}

\usepackage{scrlayer-scrpage}
\pagestyle{scrheadings}
\clearscrheadfoot
\cfoot[\thepage]{\thepage}
\pagenumbering{roman}


\makeatletter

\newcommand\fancytitlehead{
	\headingfont%
	\fontsize{80}{80}\selectfont\textcolor{black!80}{\@lastname.}\\[15pt]%
	\fontsize{60}{60}\selectfont\@ifundefined{@shorttitle}{\@title}{\@shorttitle}.%
}


\def\firstname#1{\def\@firstname{#1}}
\def\lastname#1{\def\@lastname{#1}}
\def\shorttitle#1{\def\@shorttitle{#1}}
\def\namesuffix#1{\def\@namesuffix{#1}}
\def\instrumentation#1{\def\@instrumentation{#1}}
\def\parts#1{\def\@parts{#1}}

\firstname{\relax}
\lastname{\relax}
\namesuffix{\relax}
\instrumentation{\relax}
\parts{\relax}


\def\maketitle{%
\begin{titlepage}%
	\Large%
	{\@titlehead}%
	\vfill%
	{\strut\@firstname}\\%
	{\sbseries\color{oldred}\strut\@lastname}\\%
	{\strut\@namesuffix}%
	\vfill%
	{\sbseries\@title}\\%
	{\@subtitle}\\[\baselineskip]%
	{\itshape\@instrumentation}%
	\vfill%
	{\itshape\@parts}\hspace*{\fill}\raisebox{0pt}[0pt][0pt]{\includegraphics{ees_logo}}%
\end{titlepage}%
}


\newif\iftemplate\templatetrue
\newif\ifprintreport\printreportfalse
\directlua{
scores = {
  fl1 = "Flauto I",
  fl2 = "Flauto II",
  ob1 = "Oboe I",
  ob2 = "Oboe II",
  cl1 = "Clarinetto I in B",
  cl2 = "Clarinetto II in B",
  fag1 = "Fagotto I",
  fag2 = "Fagotto II",
  ottoni = "Corno I, II in D\string\\newline Corno I, II in F\string\\newline Tromba I, II in D\string\\newline Timpani in D–A",
  vl1 = "Violino I",
  vl2 = "Violino II",
  vla = "Viola",
  b = "Bassi",
  bc = "Basso continuno",
  full_score = "Full Score"
}

last_arg = arg[\string#arg]
texio.write("Last argument: " .. last_arg)
if not (scores[last_arg] == nil) then
  tex.print("\string\\def\string\\lypdfname{" .. last_arg .. "}")
  tex.print("\string\\parts{" .. scores[last_arg] .. "}")
  if (last_arg == "full_score") then
    tex.print("\string\\printreporttrue")
  end
end
}

\@ifundefined{lypdfname}{%
  \templatefalse
  \printreporttrue
  \parts{Draft}
}{\templatetrue}

\makeatother






\begin{document}

\titlehead{\fancytitlehead}
\firstname{Friedrich Ludwig Æmilius}
\lastname{Kunzen}
\title{Ouverture nach dem Thema der Ouverture zur Zauberflöte}
\shorttitle{Ouverture}
\subtitle{(DK-Kk mu 7904.0982)}
\instrumentation{2 fl, 2 ob, 2 cl, 2 fag, 2 cor F, 2 cor B, 2 tr, trb, timp, 2 vl, vla, b}
\maketitle


\thispagestyle{empty}

\vspace*{\fill}

\raisebox{-4mm}{\includegraphics{byncsaeu}}\hspace*{1em}Wolfgang Esser-Skala, 2020

© 2020 by Wolfgang Esser-Skala. This edition is licensed under the Creative Commons Attribution-NonCommercial-ShareAlike 4.0 International License. To view a copy of this license, visit \url{http://creativecommons.org/licenses/by-nc-sa/4.0/}. 

Music engraving by LilyPond 2.18.0 (\url{http://www.lilypond.org}).\\
Front matter typeset with Source Sans Pro and Fredericka the Great.

\textit{First version, May 2020}

\vspace*{2cm}

\ifprintreport
\chapter*{Critical Report.}

This edition bases upon a contemporary print by A. Kühnel, Leipzig (1809), as provided by the Kongelige Bibliotek på Slotsholmen – Den Sorte Diamant at \url{http://img.kb.dk/ma/kunzen/kunzen_ouvzauber-m.pdf} (siglum mu 7904.0982).

In general, this edition closely follows the manuscript. Any changes that were introduced by the editor are indicated by italic type (dynamics and directions), parentheses (expressive marks) or dashes (slurs and ties). Accidentals are used according to modern conventions. Asterisks denote changes that are clarified in the detailed remarks below.

\bigskip

\footnotetext[1]{Abbreviations: fag, bassoon; Ms, manuscript; ob, oboe; r, rest; vl, violin; vla, viola.}

\begin{longtable}{ll L{10cm}}
	\toprule
	\itshape Bar & \itshape Staff & \itshape Note \\
	\midrule \endhead
	–   & b    & bass figures added by the editor \\
	67  & vla  & bar in Ms: fis′8–fis′4–fis′4–fis′4–fis′8 \\
	273 & vl 2 & 2nd quarter in Ms: bes′+g″4 \\
	\bottomrule
\end{longtable}


This edition has been compiled and checked with utmost diligence. Nevertheless, errors and mistakes cannot be totally excluded. Please report any errors and mistakes to \url{wolfgang@esser-skala.at} or create an issue or pull request on the edition’s GitHub page (https://github.com/skafdasschaf/kunzen-ouverture-zauberfloete). Your help will be greatly appreciated.

\bigskip
\textit{Salzburg, May 2020\\
Wolfgang Esser-Skala}


\cleardoublepage
\fi

\iftemplate
\includepdf[pages=-]{../out/\lypdfname.pdf}
\fi



\end{document}